\documentclass[a4paper,11pt, twocolumn]{report}
\usepackage[utf8]{inputenc}
\usepackage{natbib}
\bibliographystyle{agsm}

\setcitestyle{square}

%opening
\title{Membrane - Distributed File Backup - Literature Review}
\author{Dominic Hauton}

\begin{document}

\maketitle

\begin{abstract}
An application in which users can join a large public network of untrustworthy intelligent storage agents to trade their available file space. This will allow the user to store files safely on hostile computers, increasing availability and redundancy. The hosts will act as intelligent agents, and trade space when they communicate with one another.
\end{abstract}

\section{Introduction}
Distributed storage is a well studied and explored domain with clear advantages over bare metal. In order to implement a peer-to-peer distributed agent based storage system we must first decompose the problem into several parts and explore the advances made in those fields.

With the advent of cheap high-speed internet users are now able to use monolithic cloud services to backup data. These tend to be expensive for anything but small amounts of storage and many people have expressed security concerns over holding their data in a data centre owned by another company. Especially if their data leaves the users country where data protection laws may be different.

Over time personal storage capacities for users have also increased. It is now common for computers to come with large amounts of hard drive space which is often not filled to capacity. The project proposed promises to swap this free hard drive space to back up other users’ data in exchange for their surplus space to backup your data.

To simplify the process, the system will be able to negotiate contracts of varying complexity for space allocation on another machine, in exchange for space on itself. Unlike most distributed databases and cloud storage solutions frequent down-time will be expected across devices and contracts will have to account for this.

The challenges of such a system include a way of only copying parts of a file during updates to reduce bandwidth usage. An effective way of managing where file chunks are stored. A way to encrypt the files on the remote host and a way to authenticate that the host still holds the file.

Part of the required research also includes exploring the more recent technology of intelligent agents, in which we are most interested in trust metrics. Intelligent agents will allow the software to make storage decisions based on the trust and reputation of other nodes in the network.

\section{History of Problem}

A simple file backup can be imagined as simply copying a file to another location. In order to keep the duplicated file in sync they must be compared. The program diff solves this through finding the longest common sub-sequence of bytes between files. In order to improve performance hashing, presorting into equivalence classes, merging by binary search, and dynamic storage allocation are used. \citep{hunt1976algorithm}. This allows the user to view changes and copy the file over again if required.

\subsection{Rsync}
In a networked scenario, bandwidth from source to destination is at a premium. Rsync, introduced in 1996 presents a much better solution through copying changed file chunks (deltas). \citep{tridgell1996rsync}. Rsync splits the file into shards and calculates a weak rolling checksum and strong MD4 checksum for each block that allows quick comparisons of shards along the file. When a discrepancy is found, we assume an extra byte or bytes have been added to the file. The weak checksum can be efficiency recalculated for the next offset and once there is a match, it is confirmed with the strong checksum. The new added chunk can now be transmitted. This results in a lot less data being copied than there would be with a diff file. \citep{tridgell1996rsync} This combination of weak and strong checksums has been used across multiple distributed systems including low-bandwidth file systems \citep{muthitacharoen2001low} and high performance transactional object stores. \citep{stephen2000platypus}.

Multiround Rsync improves on the rsync algorithm by allowing for more communication to lower bandwidth. Blocks of smaller and smaller sized are used to find holes in the old file in each round and the file until the minimum block size is reached and a copy occurs. \citep{multiroundrsync} This works better than standard rsync in situations where the source file has been changed in many places distributed around the file.

Rsync requires both old and new copies of a file to exist on the host system during an update. This issue has been addressed by creating an in-place rsync (ip-rsync) that uses techniques used in delta compression and Lempel-Ziv compression to move the areas of the file around. In ip-rsync file sender sends add and move commands to the destination in an order that guarantees no files will be overwritten. \citep{rasch2003place}


\subsection{Git}
Git is an improvement on Rsync as it provides both version history and minimises data transfer. To keep storage simple, a copy of the whole file is stored and a reference is put into the version history. By storing old files locally operations are fast. This is also an important distinction from other version control systems and one of the reasons why Git was chosen as an example versioning system compared to other versioning systems like SVN. The systems can continue to operate without a centralised server. To reduce file duplication all files are referenced using their SHA-1 hash. This means you can be sure the contents of the file hasn’t changed since you last read it. \citep{torvalds2010git}

Git also uses a 3 stage workflow. A working directory, where the current files are stored, a staging area and a .git directory. The staging area prepares your next commit and then it is finally committed. When the staging is complete the change is irreversibly stored. This is a good approach that will be adopted in the final software solution. It will allow incrementally finding changed files, and assessing the need for a new version number to be saved.


\subsection{Bittorrent}
The BitTorrent protocol is a mechanism for sharing files between swarms of hosts. As BitTorrent splits files into parts, users start sharing data even before they have received the full file. Each file has a SHA-1 identifier, similar to Git. \citep{qiu2004modeling}

If a user wishes to download a file from the swarm, the user downloads a metadata file from the web and locates users sharing the data using a Tracker Server, Distributed Hash Table (DHT) or Peer Exchange (PEX). \citep{cohen2008bittorrent}

A Tracker server is a centralised store of all current connected users along with how much of the file they hold. This approach is vulnerable to exploitation by authorities as all of the data about a swarm is strored on a single server and as a result cannot be used for the proposed system.

A DHT contacts other known users for information instead of a centralised server. The Mainline DHT as outlined in BEP No.5 is based on the Kademlia protocol that allows for decentralised peer discovery for a particular piece of content.

PEX is a method for two clients to share a subset of their peer lists. Coupled with DHT, PEX removes a vulnerability from the Bittorrent network by allowing fully distributed bootstrapping, tracking and peer discovery.

A DHT with a form of PEX a tried and tested way of successfully mapping and finding files on a network and will be used within the proposed project.

\subsection{Resilio}
Resilio Sync is an example of a distributed file storage system that utilises the BitTorrent protocol to automatically synchronise folders between a user’s systems. It is not a cloud backup solution and not intended as a form of off-site storage. There is no distributed file system and as a result, no redundant data block algorithm adding complexity. \citep{farina2014bittorrent}

As Resilio Sync uses DHT to transfer data, there is no central authority to manage authentication or log data access attempts. This makes it difficult to determine whether a file has been accessed by another user. \citep{farina2014bittorrent} As a result in the project the assumption will be made that everyone in the network has access to all encrypted file chunks. To access and reassemble a file, a user will be required to request all of the file chunks individually and then locally reassemble them.

\subsection{Storij}
Storj is a peer-to-peer cloud storage network which aims to allow users to store files on a decentralised platform. The platform takes special care to provide protection against Sybil attacks and other forms of fraud. \citep{Wilkinson14storja}. To store files it stores encrypted hashed shards on the network. In order to provide proof of storage it uses Merkle Audits and pre-generated audits with hash-challenges to determine whether the client still holds the required data. By adding a seed to the hash-calculation the client can enforce the workers are still in possession of the data. It prevents the client cheating a farmer through using blockchain proof-of-existence to keep both parties honest.

The most efficient form for proof of storage is through using a deterministic heartbeat. Using Feistel permutations data can be verified with $n + 2 \sqrt{n}$ challenges. Erasure encoding is added to shards to detect any minor changes to the data. This is less I/O intensive than a full heartbeat, but still allows an attacker to complete heartbeats with only a data integrity of 1/n, where n is the number of nodes holding the data.

In order to add extra protection to files, we can use erasure encoding to allow file recovery if one of our shard types is lost. This can be investigated in our software, however, as the shards are expected to change on a regular basis because of versioning, this may not be possible.

To prevent Sybil based attacks, Storij encrypts each shard segment with a different salt. This stops workers completing proof of storage on behalf of another node.


% Official hosting solutions
% Storj
% Privacy/cost concerns.

\section{Peer Admission}
The first step in designing the distributed file system is locating other peers within the swarm. This is accomplished through Peer Admission. Once the first peer is found data within the swarm can be located using a DHT which guarantees content can always be found.

\subsection{Bootstrapping}

There are two types of PTP networks which must be examined:
\begin{itemize}
 \item \emph{Asynchronous}
 \item \emph{Synchronous}
\end{itemize}

Within Synchronous networks the number of nodes on the network is constant and all of the nodes are aware of each others existence. This does not allow storage networks to scale but it does allow data to be kept private. \citep{saxena2003admission} This is the simplest and first approach that will be taken in locating nodes within Membrane.

In most current P2P systems such as Gnutella \citep{klingberg2002gnutella}, Chord and Tapestry as well cryptocurrencies such as in Bitcoin and Litecoin a bootstrapping node is contacted, which provides information about what clients are currently online. Once a bootstrapping node allows the client to find the edge of the swarm, more information can be found using peer exchange.

Within a local network we can also use Universal Plug and Play to find other nodes within the local network. This prevents an external call to a bootstrapping node and as a result is less prone to attack.

Through looking at availability metrics within Bittorrent systems \cite{neglia2007availability} determined that both trackers and DHT should be used in creating a highly available distributed storage system such as BitTorrent. DHT tends to be slower at finding new data, however it is much more reliable.

Within Membrane, I plan to use a combination of Asynchronous and Synchronous techniques. Users will try to bootstrap from their last known neighbour nodes on the network, this takes advantage of the static nature of the backup data. Only if this fails, and with the user's permission will they contact a centralised bootstrapping node. This should only happen during a first install and if the user has no referrals. Throughout the lifetime of the application the centralised bootstrapping node would ideally be replaced with a referral system.

A further extension of this, would be to allow hosts to provide a DNS name along with their IP. Users that have setup Dynamic DNS (DDNS) \citep{bound1997dynamic} would be able to locate each other without the help of a bootstrapping server.

\subsection{Peer Exchange}

When bootstrapping is complete new Peers can increase their knowledge of the network through Peer Exchange. This is used by Bittorrent to help share swarm information with other nodes. As soon as a client connects to the swarm, peer information is collected using DHT or PEX.

There are two common extension protocols called AZMP and LTEP, which send at most one message a minute when a client leaves or exits the swarm. To reduce congestion at most 50 peers can be added or removed in one PEX message. \citep{vuze2010vuze}

\subsubsection{Shard Discovery}

Bittorrent also uses the Mainline DHT to find other hosts in the network. This is a Kademlia DHT which now according to \cite{jones2015mainlinedht}, now supports 25M users. It works through assigning each node and file a 160-bit string as an ID. We can work out which node is meant to store a file metadata and crawl in the direction of the node using a hill climb algorithm. Once the metadata is stored on the host, if a host host wants to download a file, it can take the metadata on the known host to find the IP of hosts with the file.

Hosts in Membrane will store a metadata file with all of the information required for the specific account to run, including friends, encryption keys as well as local mappings for which host owns which shard and which IP belongs to each host. As a result a DHT will only be required if this initial metadata is lost and needs to be recovered. The proposed version of Membrane will make use of this metadata, but recovery can be added in further iterations of the project.

\subsubsection{Dynamic IP Address}

Dynamic IP addresses have proven to be problematic in distributed computing, as ISP typically charge more for users to have a static, unchanging IP. Bittorrent tackles this issue through using a DHT to dynamically find the IP address of the user that owns a file. This approach is robust, however, it is a complex solution to IP address resolution.

Another widely used approach is using Dynamic DNS (DynDNS) as described by \cite{bound1997dynamic} in RFC 2136. This allows a client to automatically update a nameserver with a new IP or other information, this allows clients to have a persistent addressing method for devices that change their location. This approach requires initial configuration by the user, however, it provides a reliable way to connect with a user when their IP is lost. There are several tools such as MintDNS, cURL and Iandyn that could be used to ease the development of a built in DynDNS. When setting up a relationship with another host, both an A/AAAA Address and CNAME could be provided, where the CNAME is a backup if the A/AAAA address does not work.

To resolve IP Address resolution within Membrane, I would like to take advantage of small-world networks. \citep{porter2012small}, in which the mean shortest-path between two nodes increases slowly compared to the number of notes in a network. Within a group of users in Membrane hosts are likely to share multiple first, second and third degree connections. By storing a list of IP addresses from all of the hosts. It is highly unlikely that all connections within three hops will have changed IP address. We take inspiration from ARP \citep{plummer1982ethernet}, and send broadcasts with a limited hop count in the network to see if anyone is aware of the current address of a host. The downside of this approach is that it relies on a node within your social network to be online. Measures will need to be put in place to reduce broadcast spam.

A broadcast storms runaway broadcast events, common in networks that used broadcasting for communication, particularly when areas of the network overlap. \citep{Tseng:2002:BSP:506900.506905} These can be mitigated by reducing broadcast traffic, however, Membrane will rely on broadcasts to find hosts. The first step to limiting these broadcasts, is implementing a hop count on broadcasts. This is commonly seen in routing protocols such as IPv6 \citep{deering1998internet}. The Spanning Tree Protocol (STP) as seen in IEEE 802.1d \citep*{ieee802ieee, sharma2004viking} provides loop-free routing in LANs by pruning redundant links. Topology changes are dealt with by rebuilding the tree.

Within Membrane rebuilding a Spanning Tree would be an expensive operation. If broadcasts become a problem a 'block request' system can be used, similar to that of ICMP redirects. \citep{postel1981rfc} If a node receives a duplicate broadcast message it sends a request back one hop to not send broadcasts from that source toward it for  some time. The time limit would allow allow for corrections if the network topology changes. This preventative approach and could be improved by using the full Spanning Tree implementation. As a further step it could be improved by using a DHT 'closest jump' approach if required.

\section{Data Allocation on External Nodes}

In order to store data on another node Membrane must first have permission to store files on another node. In order to make a choice we must be able to look at trust information about other nodes on the network, and negotiate and trade space once a suitable candidate has been found. These two areas have been explored in the context of Multiagent Systems (MAS) in the past. \citep{wooldridge2009introduction}

\subsection{Negotiation}
There are three main areas that are important for negotiation. Negotiation protocols, negotiation objects and the node's reasoning models. \citep{beer1999negotiation} The sophistication of the negotiation is determined by each of these and can take different forms such as auctions, argumentation and protocols in the style of a contract net. The simplest negotiation uses fixed size items, which Membrane shall be initially using for this reason. More complex negotiation allows for counter offers and stronger guarantees.

A simple negotiation protocol issues a call for proposals to a number of nodes and waits for their bids. To formalise this a Agent Communication Language (ACL) such as KQML (Knowledge Query and Manipulation Language) \citep{finin1992specification} or the more modern FIPA (Foundation for Intelligent Physical Agents) \citep{fipa2002fipa}. \cite{beer1999negotiation} tells us that within KQML the agent sending the query is expected to decide how the receiver will handle it, which places limits on negotiation. On the other hand, FIPA is newer and as a result can be more error prone.

% TODO: Discuss Negotiation Tactic we use.

\subsection{Communication}

As Membrane is a distributed system communication is key for nodes on the network to interact. It is traditional to form a protocol that agents can use to share information with each other. These protocols are often very rigid and do not allow for expansion. Formalising communication using ACLs is a more flexible approach that aims to let agent share a common understanding of the domain, and allows hosts to reason about communication themselves. Instead of using a strict protocol we shall instead take an agent-based approach to communication using ACLs.

\subsubsection{Agent Communication Language (ACL)}

On balance, we shall be using FIPA for communication. This is an ACL based on Speech Act Theory \citep{labrou1999agent}. It splits communication into a communicative act (CA), an ACL message structure and a set of communication protocols such as XML or OWL (Web Ontology Language). In FIPA communication should be rational, in that when sending a message:
\begin{itemize}
  \item The sender believes the proposition.
  \item The recipient does not already believe the proposition.
  \item The recipient will believe the proposition after the proposition.
\end{itemize}
In the case of Membrane this could be put into the context of asking another node to store data, it would only be reasonable to send another node data to be stored, if the two nodes had negotiated storage of that block between them previously.

To keep communication simple for nodes, we shall use two CAs
\begin{itemize}
  \item REQUEST
  \item INFORM
\end{itemize}
where \emph{REQUEST} expects a reply of some sort and \emph{INFORM} does not. This will enable easier implementation and can be expanded if required.

\subsubsection{Ontology}

An ontology will need to be created for negotiation. Within MAS ontologies allow agents to share a common understanding of the structure of information,

% TODO: Talk a bit more about ontology creation


\cite{noy2001ontology} gives us 5 steps for ontology creation.

\begin{enumerate}
 \item Determine the domain and scope of the ontology
 \item Consider reusing existing ontologies
 \item Enumerate important terms
 \item Define classes and their hierarchy
 \item Define class properties (slots)
 \item Define slot facets
 \item Create Instance
\end{enumerate}

Using this process a basic ontology for contract negotiation can be created. It is important to ensure the ontology goes into the right amount of detail for the requirements. An overly detailed ontology can make reasoning difficult, and an ontology that is too simple will limit expressiveness \citep{wooldridge2009introduction}.

\subsection{Trust and Reputation}

\subsection{File Transfer Across NAT}
Network Address Translation (NAT) is used extensively to solve the IP exhaustion problem, however, it also creates a lot of well documented problems for peer-to-peer communication as hosts can now potentially share IPs. In order to transfer data over NAT a NAT-Traversal Technique must be employed. There are four major techniques described by \cite{ford2005peer} that we shall explore.

% TODO: Complete methods
% Relaying
% Connection Reversal
% UDP Hole Punching
% TCP Hole Punching
% And we are using

\section{Distributed File System}
% Separate data allocation on external nodes from storage.
% This file system is designed with an expectation that storage nodes will be down at various points in time and this must be accounted for.
% This file system is only used for backup and is therefore asymmetric, unlike most other files systems out there. Primarily a write-only file system.
% Have to take into account the connections to clients are asymmetric on residential connections.
% Have to take into account that it is mostly used to backup small documents that the user finds important. Documents or family photos, not several gb databases.
% Recommended technique is to chunk files, give them a unique ID.
% Use a journal for reads and writes. Consider journal replication!
% Can use snapshotting to ensure a user has a previous version of a file.
% Use garbage collection to allow delete recovery.
% Single master keeps structure simple.
% Should we look at writing through hosts to minimise bandwidth costs on our side?
% Look at CDN.
% Delta copying in journal.
% How to deal with stale chunks on node reconnect?

\section{Authentication}
% - You need to prove your identity when retrieving data and communicating to friend.
% - Look into Public and Private keys.
% - Users may want password.

\section{Untrustworthy Hosts \& Encryption}
% - Prove to me you're still holding my file without sending the entire file over
%      - Request random bits
%      - Salted hash of file
% - What type of encryption is considered secure and fast.

\section{Conclusion}
% - Gloss over all of the key points that will be implemented in my software.

\bibliography{literature-review.bib}

\end{document}
